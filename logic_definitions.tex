\documentclass{article}
\usepackage[utf8]{inputenc}
\usepackage[portuguese]{babel}
\usepackage{amsmath}
\usepackage{amssymb}
\usepackage{geometry}
\geometry{a4paper, margin=1in}

\begin{document}

\section*{Formalização Lógica das Definições Ontológicas}

Nesta secção, apresentamos a formalização lógica das classes \texttt{LargeVenue} e \texttt{OverBookedRoom} utilizadas na ontologia. As definições baseiam-se na Lógica de Primeira Ordem (FOL) e em construtos da \textit{Description Logic} (DL) e \textit{Semantic Web Rule Language} (SWRL).

\subsection*{1. Definição de \textit{LargeVenue}}

A classe \texttt{LargeVenue} é definida como uma subclasse de \texttt{Room} que satisfaz uma restrição de cardinalidade sobre a propriedade de dados \texttt{has\_capacity}. Em particular, qualquer instância de \texttt{Room} com capacidade superior ou igual a 80 é classificada automaticamente como uma \texttt{LargeVenue}.

Em Lógica de Descrição (DL), esta definição é expressa através de um axioma de equivalência envolvendo uma restrição de domínio de dados:

\begin{equation}
\texttt{LargeVenue} \equiv \texttt{Room} \sqcap \exists \texttt{has\_capacity}.[\ge 80]
\end{equation}

Traduzindo para Lógica de Primeira Ordem (FOL), a definição estabelece que um objeto $x$ é um \texttt{LargeVenue} se, e somente se, $x$ for uma \texttt{Room} e existir um valor de capacidade $c$ associado a $x$ tal que $c \ge 80$:

\begin{equation}
\forall x (\texttt{LargeVenue}(x) \iff \texttt{Room}(x) \land \exists c (\texttt{has\_capacity}(x, c) \land c \ge 80))
\end{equation}

Onde:
\begin{itemize}
    \item $\texttt{Room}(x)$ é o predicado unário que denota que $x$ é uma sala.
    \item $\texttt{has\_capacity}(x, c)$ é o predicado binário (propriedade de dados) que relaciona a sala $x$ com o valor inteiro $c$.
\end{itemize}

\subsection*{2. Definição de \textit{OverBookedRoom}}

A classe \texttt{OverBookedRoom} representa uma situação que requer a comparação dinâmica entre dois valores de propriedades distintas: a capacidade da sala e o número de estudantes inscritos no curso agendado para essa sala. Como a expressividade padrão da OWL 2 DL não suporta diretamente comparações aritméticas entre cadeias de propriedades de objetos distintos, esta lógica é fundamentada através de uma regra SWRL (\textit{Semantic Web Rule Language}).

A regra estipula que se uma sala tem um agendamento para um curso, e o número de estudantes inscritos nesse curso excede a capacidade da sala, então a sala é classificada como \texttt{OverBookedRoom}.

Formalmente, esta lógica é expressa como uma implicação (Cláusula de Horn) em Lógica de Primeira Ordem:

\begin{multline}
\forall r, b, c, cap, stud \quad \bigg( \\
\texttt{Room}(r) \land \texttt{has\_booking}(r, b) \land \texttt{booking\_for}(b, c) \land \\
\texttt{has\_capacity}(r, cap) \land \texttt{enrolled\_students}(c, stud) \land (stud > cap) \\
\implies \texttt{OverBookedRoom}(r) \bigg)
\end{multline}

Onde:
\begin{itemize}
    \item $\texttt{has\_booking}(r, b)$ relaciona a sala $r$ com o agendamento $b$.
    \item $\texttt{booking\_for}(b, c)$ relaciona o agendamento $b$ com o curso $c$.
    \item $\texttt{enrolled\_students}(c, stud)$ associa o curso $c$ ao número de estudantes $stud$.
    \item $\texttt{has\_capacity}(r, cap)$ associa a sala $r$ à sua capacidade $cap$.
    \item $stud > cap$ é o predicado de comparação aritmética (Built-in do SWRL).
\end{itemize}

\end{document}
